\documentclass[paper=a4, fontsize=11pt]{scrartcl}

\usepackage[utf8]{inputenc}
\usepackage{fourier} % Adobe Utopia
\usepackage[french]{babel}
\usepackage{amsmath,amsfonts,amsthm}
\usepackage{relsize}
\usepackage[acronym,toc]{glossaries}
\usepackage{svg}
\usepackage{enumitem}
\usepackage[ampersand]{easylist}
\usepackage{graphicx}

\usepackage{sectsty}
\allsectionsfont{\normalfont\scshape} 

\usepackage{fancyhdr}
\pagestyle{fancyplain}
\fancyhead{}
\fancyfoot[L]{} 
\fancyfoot[C]{}
\fancyfoot[R]{\thepage}
\renewcommand{\headrulewidth}{0pt}
\renewcommand{\footrulewidth}{0pt}
\setlength{\headheight}{14.6pt}
\graphicspath{{res/}}

\usepackage{stmaryrd}
\usepackage{url}

\newcommand{\horrule}[1]{\rule{\linewidth}{#1}}
\hoffset = -0pt
\voffset = -20pt
\textwidth = 450pt
\textheight = 700pt

\title{%
    \normalfont{}
    \normalsize{}
    \textsc{École nationale supérieure d'informatique et de mathématiques appliquées de Grenoble} \\ [10pt]
    \horrule{0.5pt} \\[0.4cm]
    \huge Prototyping a massively multiplayer game server
    \horrule{2pt} \\[0.5cm]
}

\author{Yann COLINA\\
Bastien ETCHEGOYEN\\
Etienne L'HER\\
Floran NARENJI-SHESHKALANI} 

\date{\normalsize\today}

\newacronym{MMO}{MMORPG}{massively multiplayer online role playing game}
\newacronym{WoW}{WoW}{World of Warcraft}

\begin{document}

\maketitle

\iffalse
* Reflection
* Actor system/hierarchy
* API
* Diagrams
    * Actors hierarchy/messages
    * Client protocol
* Protocol (scodec)
* Handlers
* AuthServer
    * SRP6a
* WorldServer
    * RC4a
    * Actor interactions (EventStream)
* Reverse engineering difficulties
    * Big protocol, organicly built, too many features
* Sources
    * TrinityCore
    * Akka docs
    * Scala docs
    * scodec
* Scala
    * Functional
    * Immutable
    * Preferred language for Akka actors (ugly in Java)
    * Strongly typed
    * Full of syntaxic sugar
    * Ahead of it's time ('research language')
    * Home grown language (der Schweiz)

* Introduction
    * Motivation
    * TrinityCore
        * It's not "translating C++ to Akka"
        * We're doing cleanear, hopefully scalable
    * Basic behavior of WoW client
* Building blocks for a WoW server
    * Actors and why
    * Protocol (scodec)
    * Database storage
    * Networking
    * API + Reflection
* Auth
    * Protocol description
        * Sequence diagram
        * Actor diagram
        * Principle of each packet
            * with it's cryptographical application
    * Realm list building
    * Account API
    * AuthSession FSM
* Realm
    * Connection + cryptography
    * Characters management
    * World state
        * Event stream
    * Moving entities
    * Concurrency stuff
* Conclusion
    * Reverse engineering difficulties
* Annex: Why Scala, no regrets tho
* Bibliography
    

\fi

\section{Introduction}

This project was an experiment in writing an authoritative server for an online
video game.
The goal was to get hands-on experience about designing and building a
complex server from scratch, without missing any aspects of it: from basic blocks
such as networking, cryptography and concurrency to managing the world itself.

\section{World of Warcraft}

For such a purpose, we choose the \gls{WoW} video game.
In deed, writing our own video game client would have been both out of the scope
of this project and, in terms of time spent, mutually exclusive with writing the
server for it.
Moreover, the video game sector has by nature little to no available open source
game clients that would fit the purpose of this project.
Consequently, it was decided to settle on an existing video game.

With prior knowledge and additional research, \gls{WoW} was determined to be the
video game for which writing a server would be most interesting: being the most
popular and populous game of its type for the last decade, the technical aspects
were certain to be production grade. Furthermore, the protocol is well
documented and there exists very advanced open source implementations of it.

\end{document}

